% !TeX program = pdfLaTeX
\documentclass[12pt]{article}
\usepackage{amsmath}
\usepackage{graphicx,psfrag,epsf}
\usepackage{enumerate}
\usepackage{natbib}
\usepackage{textcomp}
\usepackage[hyphens]{url} % not crucial - just used below for the URL
\usepackage{hyperref}

%\pdfminorversion=4
% NOTE: To produce blinded version, replace "0" with "1" below.
\newcommand{\blind}{0}

% DON'T change margins - should be 1 inch all around.
\addtolength{\oddsidemargin}{-.5in}%
\addtolength{\evensidemargin}{-.5in}%
\addtolength{\textwidth}{1in}%
\addtolength{\textheight}{1.3in}%
\addtolength{\topmargin}{-.8in}%

%% load any required packages here
\usepackage[french]{babel}\usepackage[intoc, french]{nomencl}


% tightlist command for lists without linebreak
\providecommand{\tightlist}{%
  \setlength{\itemsep}{0pt}\setlength{\parskip}{0pt}}

% From pandoc table feature
\usepackage{longtable,booktabs,array}
\usepackage{calc} % for calculating minipage widths
% Correct order of tables after \paragraph or \subparagraph
\usepackage{etoolbox}
\makeatletter
\patchcmd\longtable{\par}{\if@noskipsec\mbox{}\fi\par}{}{}
\makeatother
% Allow footnotes in longtable head/foot
\IfFileExists{footnotehyper.sty}{\usepackage{footnotehyper}}{\usepackage{footnote}}
\makesavenoteenv{longtable}


\usepackage{float}
\floatplacement{figure}{!htb}

\begin{document}


\def\spacingset#1{\renewcommand{\baselinestretch}%
{#1}\small\normalsize} \spacingset{1}


%%%%%%%%%%%%%%%%%%%%%%%%%%%%%%%%%%%%%%%%%%%%%%%%%%%%%%%%%%%%%%%%%%%%%%%%%%%%%%

\if0\blind
{
  \title{\bf Rapport de stage}

  \author{
        Fait par Golovanova Elizaveta \\
    BDA, Université Grenoble Alpes\\
      }
  \maketitle
} \fi

\if1\blind
{
  \bigskip
  \bigskip
  \bigskip
  \begin{center}
    {\LARGE\bf Rapport de stage}
  \end{center}
  \medskip
} \fi

\bigskip
\begin{abstract}

\end{abstract}

\noindent%
{\it Keywords:} 
\vfill

\newpage
\spacingset{1.45} % DON'T change the spacing!

\newpage
\tableofcontents 
\newpage

\hypertarget{introduction}{%
\section{Introduction}\label{introduction}}

Ce rapport est consacré à l'analyse du déroulement de stage dans le
cadre de la formation en Master 1, parcours Business et analyse de
données, Faculté d'Économie de l'Université Grenoble Alpes. La première
partie de ce rapport décrira le processus de recherche de mon stage,
l'évolution de mon CV et de ma lettre de motivation, et comment ma
candidature s'était diffusée à diverses organisations. La deuxième
partie du rapport parlera de l'organisation où j'ai effectué mon stage,
notamment ses principales activités, le nombre d'employés. La troisième
partie est réservée à la description du projet auquel j'ai participé,
c'est-à-dire son objectif global et mon rôle dans sa réalisation. Dans
la quatrième partie, une de mes missions sera décrite en détail. Dans la
cinquième partie finale, les résultats de mon stage seront résumés.

\section{Recherche de stage et présentation de la structure}
\label{sec:first}

\subsection{Recherche de stage}

Dans cette section, je décrirai comment mon CV et ma lettre de
motivation ont évolué en fonction de l'expérience de recherche d'un
stage. Au départ, j'ai rédigé un CV qui ne contenait que des faits nus
sur ma formation, mon expérience de travail et mes compétences. Par
exemple, je n'ai pas décrit mes tâches dans des emplois antérieurs, ni
précisé les matières que j'ai suivies pendant mes études. Au fil du
temps, je me suis rendu compte que ces détails sont importants pour se
démarquer des autres candidats. J'ai réfléchi plus attentivement à mes
avantages, listé mes réalisations au travail, les sujets que j'ai
étudiés. J'ai également ajouté à mon CV mes qualités personnelles que
les recruteurs recherchaient pour le poste d'analyste de données.

J'ai aussi initialement rédigé une lettre de motivation individuellement
pour chaque poste, en précisant le nom de l'entreprise, le nom du
recruteur, pour que la lettre ait l'air personnelle. Cependant, cela a
pris beaucoup de temps et n'a donné aucun résultat. Comme le processus
de candidature en France est bureaucratiquement compliqué, avec le
temps, j'ai décidé de simplifier ma lettre de motivation et de la rendre
universelle.

Je tiens également à souligner la participation de l'université aux
modifications qui ont eu lieu. Avec l'aide de Sylvian Houset, les fautes
de frappe et les formulations inexactes ont été corrigées dans mon CV et
ma lettre de motivation.

Après avoir modifié mon CV et ma lettre de motivation, j'ai commencé à
recevoir des offres d'emploi, mais elles ne me convenaient pas pour
poursuivre le processus de formation. J'ai parcouru plusieurs entretiens
et noté les entreprises où je pourrais essayer d'obtenir un emploi après
la fin de ma formation. En parallèle, j'ai diffusé activement mon CV aux
entreprises de Grenoble et ses environs, y compris les offres envoyées
par les responsables de notre master, et je me suis renseignée également
sur les postes disponibles parmi les professeurs de mon université. Au
total, j'ai postulé à plus de 70 endroits pour toute la recherche.
Finalement, j'ai trouvé un stage dans le laboratoire GAEL avec Paolo
Crosetto qui a été chargé du cours d'analyse de données en R que j'ai
suivi. Ayant déjà 4 ans d'expérience dans le domaine scientifique en
Russie et en cherchant la possibilité de poursuivre mon activité dans le
niveau mondial, j'étais convaincue que ce stage s'intègre parfaitement
dans mon projet professionnel.

\subsection{Présentation de la structure de stage}

L'organisme qui m'a accueilli est l'Institut National de Recherche pour
l'Agriculture, l'alimentation et l'Environnement (INRAE). J'ai effectué
un stage dans un des laboratoires affiliés à cet organisme, à savoir au
laboratoire d'économie appliquée à Grenoble (GAEL). Il est composé d'une
quarantaine de chercheurs auxquels s'ajoutent des post-doctorants, des
doctorants et du personnel administratif et d'appui à la recherche.

\section{Missions effectuées pendant le stage}
\label{sec:third}

L'une des principales prémisses de la plupart des modèles
microéconomiques classiques est la rationalité des individus. Cependant,
dans le monde réel, on peut observer que les gens, lorsqu'ils sont
exposés à l'incertitude, essaient de minimiser cette incertitude autant
que possible. Par exemple, un investisseur peut choisir d'investir son
argent dans un compte bancaire avec un taux d'intérêt faible mais
garanti, plutôt que dans des actions, qui peuvent avoir un rendement
attendu élevé mais également comporter un risque de perte. Cet effet est
appelé aversion au risque, il caractérise la propension des personnes à
prendre telle ou telle décision face au risque. L'analyse de l'aversion
au risque joue un rôle important en économie comportementale, car elle
permet de mieux comprendre le fonctionnement de l'irrationalité humain.

La thème du projet où j'ai participé pendant mon stage était
``Méta-analyse de la validité externe des tâches d'élicitation des
risques''. Il vise à rassembler les connaissances existantes dans la
domaine de l'elicitation des risques, de mieux comprendre l'état actuel
de la validité externe des mesures qui aident d'identifier des risques
(par méta-analyse) et donner libre accès à d'autres chercheurs et
personnes intéressées en ligne et avec une base de données en constante
augmentation.

L'enjeu principal de ce projet est de repenser les tâches d'élicitation
des attitudes face au risque. C'est-à-dire qu'au cours d'expériences en
laboratoire, il est censé effectuer certaines tâches visant à identifier
l'attitude du répondant face au risque. Une question directe peut être
posée pour le déterminer, par exemple, «~Comment vous voyez-vous~?
Êtes-vous généralement une personne totalement prête à prendre des
risques ou essayez-vous d'éviter de prendre des risques~?~»
(\citet{SOEP2007}) ou il peut être la demande d'indiquer la probabilité
d'être impliqué dans un événement particulier, comme tricher à un examen
(\citet{Blais2006}). De même il est possible d'estimer les attitudes
face au risque en utilisant une tâche d'élicitation des risques (RET),
par exemple, où le répondant est invité à choisir entre deux options
plus et moins risquées (\citet{Holt2002}) ou à choisir un seul loterie
de la liste proposé (\citet{Eckel2002}). Il existe de nombreuses
variantes de ces tâches, certaines impliquent une composante visuelle
pour une perception plus facile (\citet{Hunt2005},
\citet{Crosetto2013}). Dans quelques tâches, les pertes sont implicites
(\citet{Menkhoff_Sakha_2017}), dans d'autres seulement les gains
(\citet{Eckel2002}, \citet{Holt2002}). Dans certaines tâches, il y a le
choix entre deux loteries, dont l'une est plus risquée. D'autres peuvent
offrir un choix de loterie à risque et un certain équivalent
(\citet{Menkhoff_Sakha_2017}, \citet{Csermely2014}).

Le fait est que les RETs montrent peu de corrélation avec les mesures
autodéclarées, avec le monde réel et entre eux. En termes
psychométriques, elles montrent peu de validité prédictive. Ainsi, le
résultat de l'analyse des données recueillies au cours du projet devrait
être la création d'une nouvelle tâche qui résoudrait le problème de la
faible corrélation.

Il est à noter, que le projet pré-existait à mon stage, il a été lancé
en 2019 sous la direction de Paolo Crosetto. Au début de mon stage, une
version brute du site web avec la base de données modérée et les
fonctions limités était prête, mais elle devait être entièrement
repensée. En outre, lors de la création d'un nouveau site web, j'ai pris
quelques idées conceptuelles de la version originale.

Ma mission principale était de redévelopper un site web informatif, bien
structuré, techniquement stable et jolie sur logiciel R Studio. Pour y
parvenir, j'ai dû effectuer plusieurs sous-tâches. Tout d'abord,
nombreux articles scientifiques différents basés sur des expériences en
laboratoire ont été lus. Deuxièmement, une demande de données a été
faite pour enrichir le site auprès de la communauté scientifique en
suivant deux methodes :

\begin{enumerate}
\item Paolo a envoyé le mail au ces collègues directement en utilisant la base de contact qu'il avait;
\item Les messages personnels à diffuser aux auteurs qui Paolo ne connait pas encore ont été composé. Les articles nécessaires ont été sélectionnées par définition des requets pertinentes sur Google Scholar. Les liens sur ces articles et son information principale, tel que les noms d'auteurs et d'article et l'année de publication ont été collecté en utilisant le parser en Python. Ensuite, la lecture des articles par le site Scihub et extraction des mails par analyseur en Python ont été effectué. 
\end{enumerate}

Ensuit, il fallait d'alimenter le site web d'une quantité importante de
base de données d'élicitation de risque qu'on a reçu dans une manière
unifiée et uniformisée pour toutes les expériences. Principalement, ça
veut dire de faire les calculs nécessaires pour obtenir le paramètre
d'aversion relative constante du risque (CRRA)\footnote{\url{https://en.wikipedia.org/wiki/Risk_aversion}}
pour chaque article en utilisant la méthodologie spéciale qui sera
précisé dans la partie 4.3. Il était également nécessaire d'identifier
sans ambiguïté d'autres variables, telles que le genre, l'age, le pays
dans lequel l'expérience a été menée, la ville, etc.

Finalement, en utilisant cette base des données, le site web a été
recréé avec l'ajout de nouvelles fonctionnalités en utilisant le package
Shiny. La visualisation sur le site comprenait la conception de la
structure du site, ainsi que son contenu. Par exemple, construire des
distributions pour le paramètre CRRA pour chaque élément ou chaque type
de question, calculer également des corrélations, création de cartes
géographiques, de tableaux avec des sources de données, etc.

\section{Développement d’une mission avec problématique économique et analyse}
\label{sec:fourth}

\subsection{Problématique économique}

Des tentatives pour faire une méta-analyse des résultats de diverses
expérimentations contenant de tâches d'élicilation du risque ont déjà
été faites dans la littérature (\citet{CroFil2013b},
\citet{Alserda2019}, \citet{Bokern2021}). Cependant, une analyse
détaillée des caractéristiques des tâches, ainsi que des questions
supplémentaires aux répondants sur un large échantillon de données,
n'ont pas été réalisées auparavant. La problématique économique qui
décrit une de mes missions est une recherche des caractéristiques de la
tâche d'élicitation du risque qui expliquaient le mieux l'attitude
réelle face au risque.

\subsection{Base de données}

Actuellement, la base de données du site web contient les résultats
d'expériences de 85 articles, dont 9 tâches différentes pour éliciter
des attitudes à l'égard du risque, ainsi que 6 types de questionnaires.
Au total, 20703 personnes ont participé à ces études. Il convient de
noter que les variables recueillies auprès des participants peuvent être
diffèrent d'une étude à l'autre. Par exemple, certains auteurs n'ont pas
collecté d'informations sur l'âge des répondants. Donc, pour la présente
étude, une partie des données contenant les informations les plus
complètes sur les expériences a été sélectionnée. Elle comprenait des
données provenant de trois expériences de laboratoire
(\citet{Crosetto2013}, \citet{Crosetto2016}, \citet{Frey2017}),
totalisant 4525 observations. Ces données contiennent des informations
sur les réponses à 5 tâches différentes d'identification des risques,
ainsi que des réponses à 2 types de questionnaires. Pour chaque
participant, son choix dans la RET et dans les questionnaires, son sexe
et son âge, ainsi que le pays de l'expérience sont connus. Pour mener
cette étude, 5 types de tâches ont été sélectionnés, à savoir Holt and
Laury task (HL), Eckel and Grossman task (EG), Investment game (IG),
Balloon Analog Risk Task (BART) et The bomb risk elicitation task
(BRET). Des informations détaillées sur les tâches considéré sont
décrites dans la subsection ``Types de tâches d'élicitation du risque''.
Le tableau 1 ci-dessous fournit des statistiques descriptives pour cette
base de données.

Tableau 1 - Statistiques descriptives de la base de données.

\begin{longtable}[]{@{}lll@{}}
\toprule()
Variables & & n \\
\midrule()
\endhead
Genre & Femme & 2692 \\
& Homme & 1833 \\
Age & 18-25 & 3047 \\
& 26-35 & 1447 \\
& 36-44 & 23 \\
& 45-54 & 5 \\
& plus que 55 & 3 \\
Attitude à l'égard du risque & Réticent au risque & 3775 \\
& Risque neutre & 225 \\
& Preneur de risques & 525 \\
Types de taches & HL & 1399 \\
& IG & 86 \\
& EG & 88 \\
& BRET & 1447 \\
& BART & 1505 \\
Pays & Germany & 3130 \\
& Switzerland & 1395 \\
\bottomrule()
\end{longtable}

Je tiens d'ajouter quelques mots de la modification de la base de
données initiale. Avant mon stage, elle était composée d'environ 55
articles contenant 6 types de RET et 2 types de questionnaires. Il
contenait des données telles que le choix du répondant à la loterie
et/ou dans le questionnaire SOEP et/ou DOSPERT, son sexe, son âge, s'ils
étaient indiqués dans l'article. Pendant le stage, des informations sur
le pays de l'expérience ont été ajoutées pour chacune des 85 articles.
En travaillant de cette étude, j'ai compilé d'autres variables, décrites
dans la section ``Variables pour l'étude''.

\subsection{Types des tâches d'élicitation du risque}

Examinons de plus près les types des tâches choisi pour cette étude.
L'une des RET les plus populaires est \emph{Multiple price list} de Holt
and Laury (HL, \citet{Holt2002}). Les sujets sont confrontés à une série
de choix entre des paires de loteries, l'option A étant plus sûre que
l'option B (Tableau 2). Espérance mathématique est commun à tous les
choix, et les paires de loterie sont classées par valeur attendue
croissante. Les sujets font un choix pour chaque paire de loteries et
doivent à un moment donné passer à l'option risquée. Le point de
commutation capte l'aversion au risque du sujet. Un sujet neutre au
risque devrait commencer par l'option A et passer à B à partir du
cinquième choix. À la fin de l'expérience, une ligne est choisie au
hasard pour le paiement, et la loterie choisie est jouée pour déterminer
le gain. Il convient de noter que si le répondant passait de la loterie
A à la loterie B, puis vice versa, ces cas étaient supprimés de la base
de données, car le répondant était considéré comme incohérent.

Tableau 2 - La liste de prix multiple de Holt and Laury

\hfil \includegraphics[width=0.6\textwidth,height=0.3\textheight]{/Users/elizavetagolovanova/Dropbox/METARET/METARET_github/Report de stage/HL.png}
\hfil

Source : \citet{Holt2002}

Une loterie pas peu connue est aussi \emph{Ordered lottery selection} de
Eckel and Grossman (EG, \citet{Eckel2002}). Les sujets choisissent une
loterie parmi un ensemble de 5 loteries caractérisées par une valeur
attendue croissante linéairement ainsi qu'un écart type plus grand, avec
la probabilité de chaque option étant fixée à 50 \% (Tableau 3). Un
participant neutre au risque devrait choisir la loterie 5, car elle
donne la valeur attendue la plus élevée.

Tableau 3 - La sélection de loteries triées de Eckel and Grossman

\hfil \includegraphics[width=0.6\textwidth,height=0.3\textheight]{/Users/elizavetagolovanova/Dropbox/METARET/METARET_github/Report de stage/EG.png}
\hfil

Source : \citet{Eckel2002}

Le prochain type de tâche s'appelle \emph{The investment game} (IG,
\citet{Gneezy1997}). Les sujets doivent décider comment répartir une
montant disponible entre un option sûr de le garder et un investissement
risqué qui donnera plusieurs fois le montant investi ou zéro avec une
probabilité égale. Dans ce cas, un sujet neutre au risque devrait
investir toute sa montant disponible.

La RET suivant est \emph{The bomb risk elicitation task} (BRET). C'est
une tâche d'identification visuelle des risques en temps réel présentée
par \citet{CroFil2013b}. Les sujets se voient proposer un champ avec des
boîtes mesurant 10 × 10. Ils sont informés que 99 boîtes sont vides et
une seule contient une bombe programmée pour exploser. Pour chaque boîte
ouverte, le répondant reçoit de l'argent. Sous le champ se trouvent les
boutons ``Démarrer'' et ``Arrêter''. A partir du moment où le sujet
appuie sur ``Démarrer'', une boîte est automatiquement collecté chaque
seconde, en commençant par le coin supérieur gauche du carré. Mais si
une boîte avec une bombe a été ouverte, il ne devient connu qu'après
avoir appuyé sur le bouton d'arrêt. Alors, toutes les accumulations sont
annulées.

\begin{figure}
\centering
\includegraphics[width=0.6\textwidth,height=0.3\textheight]{/Users/elizavetagolovanova/Dropbox/METARET/METARET_github/Report de stage/BRET.png}
\caption{L'interface BRET après 32 secondes}
\end{figure}

Source : \citet{CroFil2013b}

Encore une tâche considérée similaire à BERT est \emph{The Balloon
Analogue Risk Task} (BART, \citet{Hunt2005}). Dans cette tâche, le
participant se voit présenter un ballon et se voit offrir la possibilité
de gagner de l'argent en gonflant le ballon en cliquant sur un bouton.
Chaque clic provoque le gonflement progressif du ballon et l'ajout
d'argent à un compteur jusqu'à un certain seuil, à partir duquel le
ballon est surgonflé et explose. Les participants ne sont pas informés
des points d'arrêt des ballons. Ainsi, chaque pompe confère un plus
grand risque, mais aussi une plus grande récompense potentielle. Si le
participant choisit d'encaisser avant que le ballon n'explose, il
récupère l'argent gagné pour cette jeu, mais si le ballon explose, les
gains pour cet essai sont perdus.

Tous les types de RETs decrit ci-dessus aident à classifier les
répondants en fonction de leur attitude à l'égard du risque. Dans le
cadre du projet, on a supposé que les participants ont la même fonction
de préférence pour le risque \(u(x) = x^r\). Ainsi, l'attitude face au
risque peut être résumée par le coefficient d'aversion relative au
risque r. Considérons la loterie proposée par \citet{Holt2002}. Pour
chaque moment de changement des préférences de la loterie A à la loterie
B, il est possible de déterminer l'intervalle où tombe la valeur de ce
paramètre.

Par exemple, si le répondant a choisi sur la quatrième ligne la loterie
B, alors la solution de l'équation
\(0.4*4^r + 0.6*3.2^r = 0.4*7.7^r + 0.6*0.2^r\) sera le seuil supérieur
de l'intervalle et la solution de
\(0.3*4^r + 0.7*3.2^r = 0.3*7.7^r + 0.7*0.2^r\) sera celui de inférieur.

\begin{figure}
\centering
\includegraphics[width=0.6\textwidth,height=0.4\textheight]{/Users/elizavetagolovanova/Dropbox/METARET/METARET_github/Report de stage/BART.png}
\caption{L'interface BART après 5 pompes}
\end{figure}

Source : \url{https://timo.gnambs.at/research/bart}

Pour tenir compte d'une éventuelle erreur de mesure, lors de la mise à
jour du site, pour ce répondant une des valeurs de CRRA distribué
uniformément est implémentée avec une valeur minimale de 0.32 et une
valeur maximale de 0.59. Comme ça les valeurs appropriées ont été
calculées pour tous les participants de ce type de tâche. Dans le point
de commutation sur la toute première loterie, seul le seuil inférieur
est utilisé, et dans le cas de la dixième - seul le seuil supérieur. Si
le répondant a choisi la loterie sûre tout le temps, alors de tels cas
ont été exclus.

De manière similaire il est possible de calculer le parametre r pour
\emph{Ordered lottery selection} de Eckel and Grossman. Pour chaque
loterie sélectionnée, les seuils supérieur et inférieur de l'intervalle
sont calculés. Par exemple, si la loterie 3 est sélectionné, alors le
seuil supérieur sera la solution d'équation suivant
\(8^r + 2^r = 10^r + 1^r\).

Considérons un exemple pour montrer comment calculer le r pour
\emph{Investment Game} RET. Soit la montant disponible est de 50, et
lors de l'investissement, le montant peut augmenter de 3,5 fois avec une
probabilité de 50\%, alors: \((50 - x)^r + 0.5 * (3 * x)^r\). Pour
obtenir équation dépendant du choix d'un répondant, il faut maximiser
cette fonction et puis la résoudre par rapport à l'appétit pour le
risque (r). Le montant de la montant disponible investie est mis dans
l'expression résultante pour chaque répondant.

La formulation du BRET et BART tâches permet de mesurer l'attitude face
au risque du répondant sans recourir à des formules complexes. Ainsi,
dans le cas du BRET il suffit de compter le nombre de boîtes ouvertes et
de les diviser par le nombre de celles restantes. Ainsi, un répondant
neutre au risque ouvrira 50 boîtes de 100.

\subsection{Questionnaires}

Dans cette étude, deux types de questionnaires ont été considérés. Le
premier type est une question sur la volonté de prendre des risques du
questionnaire SOEP (\citet{SOEP2007}) suivant «~Comment vous
voyez-vous~? Êtes-vous généralement une personne totalement prête à
prendre des risques ou essayez-vous d'éviter de prendre des risques~?~»
La question est répondue sur une échelle de 0 à 10.

Le deuxième type est le questionnaire DOSPERT. C'est une échelle
psychométrique qui évalue la prise de risque dans cinq domaines :
décisions financières (décomposé ensuite en jeux d'argent et
investissement), santé/sécurité, décisions récréatives, éthiques et
sociales, à l'aide de 40 questions au total. Les répondants évaluent la
probabilité qu'ils s'engagent dans des activités à risque spécifiques à
leur domaine en utilisant échelle de 1 à 7. Les réponses à ces questions
sont ensuite pondérées par chaque domaine et additionnées pour former
une seule série agrégée. Pour générer une version courte de l'échelle
avec des questions qui seraient interprétables par un plus large
éventail de répondants dans différentes cultures, les 40 items de
l'échelle originale (\citet{Weber2002}) ont été réduits à 30 items
(\citet{Blais2006}), ce qui a été utilisé dans le rapport actuel.

\subsection{Variables pour l'étude}

Il a été mentionné au-dessus que la base de données contient des données
sur les répondants, à savoir le sexe, l'âge et le pays de participation
à l'expérience. Pour répondre au problème posé, il est nécessaire de
mettre en évidence certaines variables qui permettront de décrire les
tâches. Et donc, 8 nouvelles variables ont été inventées : is\_visual,
is\_price\_list, probabilities\_change, stakes et variables dummy pour
chaque type de tâche - BRET, BART, EG, IG, où HL est la référence.
Ci-dessous il y a un tableau de toutes les variables explicatives pour
cette étude.

Tableau 4 - Les variables explicatives

\begin{longtable}[]{@{}
  >{\raggedright\arraybackslash}p{(\columnwidth - 4\tabcolsep) * \real{0.2095}}
  >{\raggedright\arraybackslash}p{(\columnwidth - 4\tabcolsep) * \real{0.5238}}
  >{\raggedright\arraybackslash}p{(\columnwidth - 4\tabcolsep) * \real{0.2667}}@{}}
\toprule()
\begin{minipage}[b]{\linewidth}\raggedright
Nom de variable
\end{minipage} & \begin{minipage}[b]{\linewidth}\raggedright
Commentaire
\end{minipage} & \begin{minipage}[b]{\linewidth}\raggedright
Valeurs de variable
\end{minipage} \\
\midrule()
\endhead
age & Age de répondant & variable continue \\
is\_femme & Genre de répondant & 1 - Femme \\
& & 0 - Homme \\
germany & Si le pays d'expérimentation est l'Allemagne & 1 - Germany \\
& & 0 - Switzerland \\
is\_visual & La tâche a-t-elle un aspect visuel ? & 1 - BART ou BRET \\
& & 0 - les autres RETs \\
is\_price\_list & La tâche contient-elle une liste de loteries~? & 1 -
HL ou EG \\
& & 0 - les autres RETs \\
probabilities\_change & La tâche implique-t-elle des probabilités
variables~? & 1 - HL, BART ou BRET \\
& & 0 - les autres RETs \\
stakes & Rémunération de la neutralité au risque & 2.5 - BRET,
\citet{Crosetto2013} \\
& & 5 - BRET, \citet{Crosetto2016} \\
& & 1.6 - BART \\
& & 6 - EG \\
& & 4.86 - HL, \citet{Crosetto2016} \\
& & 5.82 - HL, \citet{Frey2017} \\
& & 5 - IG \\
BRET & Type de loterie & 1 - BRET \\
& & 0 - HL \\
BART & Type de loterie & 1 - BART \\
& & 0 - HL \\
EG & Type de loterie & 1 - EG \\
& & 0 - HL \\
IG & Type de loterie & 1 - IG \\
& & 0 - HL \\
\bottomrule()
\end{longtable}

Les données pour toutes les variables ont été tirées des articles
respectifs. Variable \emph{stakes} a été calculée manuellement de la
manière suivante. La stratégie du répondant neutre au risque lorsqu'il
répond à la tâche BRET est d'arrêter le jeu à exactement la moitié des
boîtes ouvertes. Étant donné que l'emplacement de la bombe n'est révélé
qu'après avoir appuyé sur le bouton ``Arrêter'', la probabilité de ne
pas rencontrer la bombe est de \(0.5\). Pour l'ouverture de chaque
boîte, une récompense en espèces est donnée. Dans l'article de
\citet{Crosetto2013}, il était de \(0.1\) euro et dans l'article de
\citet{Crosetto2016}, il était de \(0.2\) euro. La formule de calcul du
gain neutre au risque est donc \(0.1 * 100 / 2 * 0.5\) et
\(0.2 * 100 / 2 * 0.5\) respectivement pour chaque article. De même,
dans la tâche BART, le neutre au risque choisirait également la
stratégie du demi-maximum. Pour chaque gonflage du ballon, \(0.005\)
euros sont versés. Au total, il est proposé de gonfler \(10\) ballons.
Ainsi, pour le jeu, neutre au risque reçoit une récompense
\(128 / 2 * 0.005 * 10 * 0.5\). Dans la tâche EG (\citet{Crosetto2016})
neutre au risque choisira la loterie avec le gain le plus élevé,
c'est-à-dire la dernière ligne. Ainsi, son gain sera de \(12 * 0.5\).
Dans la HL tâche, le répondant neutre au risque passera à l'option
risqué à partir de la cinquième ligne. Selon les règles, une seule
loterie sur dix sera joué à la fin de la partie. Trouvons tous les gains
attendus pour chaque loterie en s'appuyant sur le tableau 2 (tableau 5).
Dans la colonne de droite, les loteries qui ont été choisies par le
répondant neutre au risque ont listé. Résumons les gains. L'occurrence
de chacun des 10 évènements est également probable, on peut donc diviser
la valeur résultante par 10 pour trouver l'espérance mathématique
moyenne de gagner. \newpage Tableau 5 - Les calculs des gains attendus

\begin{longtable}[]{@{}
  >{\raggedright\arraybackslash}p{(\columnwidth - 6\tabcolsep) * \real{0.0921}}
  >{\raggedright\arraybackslash}p{(\columnwidth - 6\tabcolsep) * \real{0.3421}}
  >{\raggedright\arraybackslash}p{(\columnwidth - 6\tabcolsep) * \real{0.3421}}
  >{\raggedright\arraybackslash}p{(\columnwidth - 6\tabcolsep) * \real{0.2237}}@{}}
\toprule()
\begin{minipage}[b]{\linewidth}\raggedright
\end{minipage} & \begin{minipage}[b]{\linewidth}\raggedright
Espérance de loterie A
\end{minipage} & \begin{minipage}[b]{\linewidth}\raggedright
Espérance de loterie B
\end{minipage} & \begin{minipage}[b]{\linewidth}\raggedright
Options choisi
\end{minipage} \\
\midrule()
\endhead
1 & 3.28 & 0.95 & 3.28 \\
2 & 3.36 & 1.7 & 3.36 \\
3 & 3.44 & 2.45 & 3.44 \\
4 & 3.52 & 3.2 & 3.52 \\
5 & 3.6 & 3.95 & 3.95 \\
6 & 3.68 & 4.7 & 4.7 \\
7 & 3.76 & 5.45 & 5.45 \\
8 & 3.84 & 6.2 & 6.2 \\
9 & 3.92 & 6.95 & 6.95 \\
10 & 4 & 7.7 & 7.7 \\
Total & & & 48.55 \\
\bottomrule()
\end{longtable}

Ainsi, la valeur d'espérance pour la tâche HL pour l'article
\citet{Crosetto2016} est de \(4.86\). Une procédure similaire a été
effectuée pour l'article de \citet{Frey2017}, la valeur de gain
résultante est de \(5.82\). Lors de la participation à la tâche IG, la
personne neutre au risque doit investir la totalité de son montant
disponible. Donc, pour l'article de \citet{Crosetto2016}, ce serait
\(4 * 2.5 * 0.5\).

\subsection{Méthodologie}

Afin d'identifier les caractéristiques des tâches qui décrivent le mieux
les attitudes face au risque, il est nécessaire de déterminer ce qui
reflète réellement la perception du risque pour le répondant. En tant
que reflet de la réalité, on peut considérer les réponses aux questions
directes sur soi, c'est-à-dire SOEP et DOSPERT. Ainsi, le critère de
qualité pour caractériser la tâche sera la distance entre le paramètre
CRRA et les réponses aux questions.

La corrélation entre les réponses de questionnaires est de 0.43, il est
donc logique de considérer deux modèles avec chaque questionnaire
séparément. Pour comparer deux séries, il faut les mettre à des valeurs
comparables. DOSPERT varie de 1 à 7 et le paramètre CRRA se situe
quelque part entre -2 et 2.5. Il faut ``tendre'' les valeurs des
paramètre R sur l'échelle de DOSPERT. J'ai fait la transformation
suivant le formule
\(\frac{r - r_{min}}{r_{max} - r_{min}} * (t_{max} - t_{min}) + t_{min}\),
ou \(r\) - valeur de CRRA, \(r_{min}\), \(r_{max}\) - valeurs du CRRA
minimale et maximale; \(t_{max}\), \(t_{min}\) - valeurs de SOEP ou
DOSPERT maximale et minimale.

Si nous voulons minimiser la différence entre les séries, on a besoin
que les points de neutralité du risque de deux distributions coïncident.
Pour CRRA = 1, la valeur d'une nouvelle échelle de y
(y\_r\_DospertScale) est égale à 4.83352. On sait aussi que le répondant
neutre au risque obtient 4 sur l'échelle DOSPERT réelle. Soustrayons la
valeur 0.83352 des valeurs de la série obtenu. Puisque nous avons
soustrait la valeur, la queue droite a été mise à zéro dessus. On peut
tronquer la distribution sur la queue droite pour ne faire correspondre
que les réponses réelles. Après une telle transformation, 4465
observations sont restées dans l'échantillon. Maintenant on peut
comparer ces deux distributions (Figure 3).

\begin{figure}
\centering
\includegraphics{Report-de-stage_files/figure-latex/distdospert-1}
\caption{Comparaison de deux distributions pour l'échelle DOSPERT}
\end{figure}

J'ai répété la procédure avec mise à l'échelle et avec un décalage pour
superposer les échelles. La neutralité au risque parmi les répondants de
cet échantillon correspond à la valeur 7.542214. Pour répondre à SOEP,
c'est le chiffre 5. Donc, j'ai soustrait la valeur de 2.542214 de la
nouvelle échelle de y (y\_r\_SoepScale). De manière similaire à la
procédure avec DOSPERT, on coupe une partie de la queue droite de la
distribution de SOEP. Après une telle transformation, 4429 observations
sont restées dans l'échantillon. Le graphique 4 montre ces deux
distributions.

\begin{figure}
\centering
\includegraphics{Report-de-stage_files/figure-latex/distsoep-1}
\caption{Comparaison de deux distributions pour l'échelle SOEP}
\end{figure}

Couper la queue droite aide également à supprimer les valeurs aberrantes
de la distribution. La limitation initiale du paramètre CRRA retenu pour
l'étude, était conditionnelle, limitant les personnes ayant une forte
propension à prendre des risques. Il n'y avait aucune restriction sur
les échelles DOSPERT et SOEP, bien que les répondants obtenant de 6 à 7
sur l'échelle DOSPERT et de 7 à 10 sur l'échelle SOEP soient des
risquephiles prononcés. Ainsi, pour la pureté de l'expérience, il est
logique de filtrer les réponses également pour les deux questionnaires.

On dénote par la variable y\_gap\_Dospert la différence entre les séries
CRRA avec une nouvelle échelle de 1 à 7 et DOSPERT, et par la variable
y\_gap\_Soep la différence entre les séries CRRA avec une nouvelle
échelle de 0 à 10 et SOEP. Prenons le valeur absolue des valeurs
obtenues, puisque nous ne nous intéressons qu'à l'amplitude de l'écart,
et non à la direction.

\subsection{Resultats}

Au cours de cette étude, on a tenté de construire un modèle de
régression linéaire pour chaque type de questionnaire sur les certaines
variables explicatives choisies pour l'analyse :

\(y_i = c_i + beta_1 * is\_female_i+ beta_2 * age_i + beta_3 * is\_visual_i + beta4 * is\_price\_list_i + beta_5 * probabilities\_change_i + beta_6 * stakes_i + beta_7 * germany_i + epsilon_i\)
, où \(y_i\) est \(y\_gap\_Dospert_i\) et \(y\_gap\_Soep_i\).

Pour la comparabilité des résultats, un échantillon a été créé à partir
duquel les valeurs aberrantes de SOEP et de DOSPERT ont été exclues. Il
contenait 4213 observations. Pour vérifier la robustesse des résultats,
les modèles avec les échantillons complets pour chaque type de
questionnaire sont également donnés.

Pour chaque variable à gauche se trouve le résultat sur son échantillon
complet, et à droite - le résultat sur un échantillon réduit pour
comparer deux variables. On peut voir que le coefficient de
détermination pour toutes les régressions est assez faible, et explique
environ 10\% de la variation pour l'écart entre CRRA et DOSPERT et
environ 7\% de la variation pour celui-ci entre CRRA et SOEP. Il est à
noter que sauf la constante de la régression sur y\_gap\_Dospert, les
résultats sont stables, car ils ne changent pas de sens de l'influence
ou de signification.

Selon les résultats obtenus, si le répondant est une femme, alors
l'écart est plus faible pour les deux types de questionnaires. Egalement
pour les tâches BRET, BART et HL, où les probabilités changent, la
différence avec la réalité est moindre. L'écart est plus important
lorsque la tâche est une liste des loteries (HL et EG). Par rapport à la
Suisse, l'Allemagne présente en moyenne un écart plus important pour
DOSPERT. On peut également conclure que l'âge n'affecte pas l'importance
de l'écart.

Tableau 6 - Résultats de l'évaluation des régressions linéaires sur les
caractéristiques des RETs et des répondants.

\begin{table}[!htbp] \centering 

  \label{} 
\small 
\begin{tabular}{@{\extracolsep{-5pt}}lcccc} 
\\[-1.8ex]\\[-1.8ex] & \multicolumn{2}{c}{y\_gap\_Dospert} & \multicolumn{2}{c}{y\_gap\_Soep} \\ 
\hline \\[-1.8ex] 
 is\_female & $-$0.044$^{***}$ & $-$0.076$^{***}$ & $-$0.509$^{***}$ & $-$0.506$^{***}$ \\ 
  & (0.015) & (0.014) & (0.053) & (0.055) \\ 
  age & $-$0.001 & $-$0.001 & $-$0.0001 & $-$0.001 \\ 
  & (0.002) & (0.002) & (0.007) & (0.008) \\ 
  is\_visual & 0.744$^{***}$ & 0.457$^{***}$ & $-$0.627$^{**}$ & $-$0.687$^{**}$ \\ 
  & (0.085) & (0.082) & (0.301) & (0.320) \\ 
  is\_price\_list & 0.126$^{*}$ & 0.212$^{***}$ & 1.073$^{***}$ & 1.096$^{***}$ \\ 
  & (0.076) & (0.070) & (0.267) & (0.272) \\ 
  probabilities\_change & $-$0.375$^{***}$ & $-$0.404$^{***}$ & $-$0.639$^{***}$ & $-$0.615$^{***}$ \\ 
  & (0.055) & (0.050) & (0.193) & (0.195) \\ 
  stakes & 0.199$^{***}$ & 0.111$^{***}$ & $-$0.268$^{***}$ & $-$0.291$^{***}$ \\ 
  & (0.014) & (0.015) & (0.049) & (0.059) \\ 
  germany & 0.092$^{***}$ & 0.079$^{***}$ & $-$0.053 & $-$0.027 \\ 
  & (0.017) & (0.016) & (0.061) & (0.063) \\ 
  Constant & $-$0.245$^{**}$ & 0.238$^{**}$ & 4.593$^{***}$ & 4.698$^{***}$ \\ 
  & (0.103) & (0.104) & (0.363) & (0.407) \\ 
 \hline \\[-1.8ex] 
Observations & 4,465 & 4,213 & 4,429 & 4,213 \\ 
R$^{2}$ & 0.098 & 0.091 & 0.069 & 0.064 \\ 
Adjusted R$^{2}$ & 0.096 & 0.090 & 0.067 & 0.062 \\ 
\hline \\[-1.8ex] 
\textit{Note:}  & \multicolumn{4}{r}{$^{*}$p$<$0.1; $^{**}$p$<$0.05; $^{***}$p$<$0.01} \\ 
\end{tabular} 
\end{table}

Les résultats contradictoires pour les deux questionnaires se trouvent
dans la présence d'une composante visuelle et les \emph{stakes}, car ils
affectent l'écart dans les données considérées de manière différente.

On peut voir que les caractéristiques choisies pour le modèle sont
largement déterminées par le type de tâche lui-même, et il est possible
que la faible variabilité des caractéristiques au sein de chaque tâche
séparément ne permette pas d'obtenir des résultats plus univoques.
Testons cette hypothèse en ne laissant que les variables de contrôle sur
les répondants, en ajoutant des variables dummy pour chaque RET. Comme
variable de référence, on prend HL.

\(y_i = c_i + beta_1 * is\_female_i+ beta_2 * age_i + beta_3 * germany_i + beta_4 * BRET_i + beta_5 * BART_i + beta_6 * EG_i + beta_7 * IG_i + epsilon_i\)
, où \(y_i\) est \(y\_gap\_Dospert_i\) et \(y\_gap\_Soep_i\).

Les deux premières colonnes du tableau 7 indiquent les estimations des
coefficients pour les modèles obtenus. Il convient de noter que le
coefficient de détermination des deux modèles a augmenté. On peut voir
que l'écart diminue si le type de tâche est BART ou BRET. Essayons de
déterminer laquelle des deux tâches nous permet de réduire le plus
l'écart. Pour ce faire, prenons la tâche BART comme référence (colonnes
3 et 4). Écart pour la variable y\_gap\_Soep est moindre avec BART et
pour la variable y\_gap\_Dospert avec BRET.

Si l'on part du fait que les réponses à l'enquête DOSPERT contiennent
plus d'informations, et que les modèles pour DOSPERT ont un coefficient
de détermination plus élevé, alors on peut créer une hiérarchie de
préférences pour l'utilisation des tâches selon le critère de réduction
de l'écart avec le questionnaire DOSPERT. Donc, la liste de préférences
ressemble à ceci :

\begin{enumerate}
\def\labelenumi{\arabic{enumi}.}
\tightlist
\item
  BART
\item
  BRET
\item
  HL
\item
  IG
\item
  EG
\end{enumerate}

Les tâches visuelles viennent en premier, car elles sont probablement
plus intuitives. Ils proposent également un grand nombre d'options au
choix (100 pour BRET et 128 pour BART).

À l'avenir, il serait intéressant d'étudier séparément les tâches
visuelles pour différentes caractéristiques, par exemple pour un
\emph{stake}, cependant, la variabilité insuffisante des ce variable ne
le permet pas pour le moment.

Tableau 7 - Les résultats de l'évaluation des régressions linéaires sur
les types de tâches et les caractéristiques des répondants.

\begin{table}[H] \centering 

  \label{} 
\small 
\begin{tabular}{@{\extracolsep{-5pt}}lcccc} 
\\[-1.8ex]\\[-1.8ex] & y\_gap\_Soep & y\_gap\_Dospert & y\_gap\_Soep & y\_gap\_Dospert \\ 
\hline \\[-1.8ex] 
 is\_female & $-$0.518$^{***}$ & $-$0.070$^{***}$ & $-$0.518$^{***}$ & $-$0.070$^{***}$ \\ 
  & (0.055) & (0.014) & (0.055) & (0.014) \\ 
  age & $-$0.004 & 0.001 & $-$0.004 & 0.001 \\ 
  & (0.008) & (0.002) & (0.008) & (0.002) \\ 
  germany & 0.032 & 0.038$^{**}$ & 0.032 & 0.038$^{**}$ \\ 
  & (0.066) & (0.017) & (0.066) & (0.017) \\ 
  BRET & $-$0.944$^{***}$ & $-$0.043$^{**}$ &  &  \\ 
  & (0.078) & (0.020) &  &  \\ 
  BART & $-$0.523$^{***}$ & $-$0.253$^{***}$ & 0.420$^{***}$ & $-$0.210$^{***}$ \\ 
  & (0.065) & (0.017) & (0.077) & (0.020) \\ 
  HL &  &  & 0.944$^{***}$ & 0.043$^{**}$ \\ 
  &  &  & (0.078) & (0.020) \\ 
  EG & 0.517$^{***}$ & 0.452$^{***}$ & 1.461$^{***}$ & 0.495$^{***}$ \\ 
  & (0.196) & (0.050) & (0.194) & (0.049) \\ 
  IG & $-$0.286 & 0.128$^{**}$ & 0.657$^{***}$ & 0.171$^{***}$ \\ 
  & (0.197) & (0.050) & (0.196) & (0.050) \\ 
  Constant & 3.541$^{***}$ & 0.655$^{***}$ & 2.597$^{***}$ & 0.612$^{***}$ \\ 
  & (0.202) & (0.051) & (0.196) & (0.050) \\ 
 \hline \\[-1.8ex] 
Observations & 4,213 & 4,213 & 4,213 & 4,213 \\ 
R$^{2}$ & 0.065 & 0.104 & 0.065 & 0.104 \\ 
Adjusted R$^{2}$ & 0.064 & 0.102 & 0.064 & 0.102 \\ 
\hline \\[-1.8ex] 
\textit{Note:}  & \multicolumn{4}{r}{$^{*}$p$<$0.1; $^{**}$p$<$0.05; $^{***}$p$<$0.01} \\ 
\end{tabular} 
\end{table}

\subsection{Limitations et difficultées}

Les limitations de cette étude comprennent le manque d'intégralité des
informations de la base de données pour certaines tâches. C'est à dire,
DOSPERT et SOEP n'ont pas été interrogés pour toutes les tâches. Par
exemple, l'information d'\emph{Ordered lottery selection} de Eckel and
Grossman avec la possibilité de pertes n'a pas pu être utilisé car il ne
contient que des données pour les SOEP questionnaires. De plus, d'autres
types de questions, comme BIS ou AuditS, n'ont pas été posées lors de
certaines RETs. À l'avenir, lorsqu'une quantité suffisante de données
expérimentales sera réunie, cette lacune sera comblée.

Il serait également intéressant d'étudier les différentes
caractéristiques d'un même type de tâche afin d'affiner le design de
l'expérience. Cependant, avec la base de données actuelle, cela n'est
pas encore possible à mettre en œuvre.

\section{Conclusion}

Le stage au laboratoire GAEL m'a donné l'opportunité de poursuivre mes
activités scientifiques, que j'exerce depuis quatre ans. Je me suis
familiarisé avec le monde de l'économie comportementale, où j'ai pu
appliquer mes connaissances et mes compétences en programmation. Pendant
les quatre mois du stage, j'ai réussi à accomplir de nombreuses tâches
complexes pour la mise en œuvre du projet. Tout d'abord, j'ai collecté
et traité beaucoup de données provenant de diverses expériences.
Deuxièmement, j'ai repensé le site web avec visualisation et analyse de
la base de données. Troisièmement, dans ce rapport, j'ai analysé les
tâches actuelles et leurs caractéristiques pour la conception future de
l'expérience.

La réalisation de ce travail a aussi amélioré mes compétences en
programmation, car je peux désormais créer des sites web basés sur le
logiciel R à l'aide du package Shiny. Grâce à la tâche de collecter des
informations sur les courriers des auteurs à partir d'Internet, j'ai
appris à collecter des textes à partir de formats PDF en utilisant
Python.

De plus, grâce à ce stage, j'ai régulièrement échangé avec des personnes
intéressantes et cultivés, ce qui a été un plaisir, et m'a également
permis d'améliorer mon niveau de français.

En résumé, l'expérience acquise a élargi la liste de mes acquis et
compétences, ainsi que mes intérêts scientifiques. Le stage en
laboratoire m'a permis de faire connaître la culture du travail, de
nouer de nouveaux contacts et d'acquérir ma première expérience
professionnelle en France.

\bibliographystyle{agsm}
\bibliography{bibliography.bib}


\end{document}
