% !TeX program = pdfLaTeX
\documentclass[12pt]{article}
\usepackage{amsmath}
\usepackage{graphicx,psfrag,epsf}
\usepackage{enumerate}
\usepackage{natbib}
\usepackage{textcomp}
\usepackage[hyphens]{url} % not crucial - just used below for the URL
\usepackage{hyperref}

%\pdfminorversion=4
% NOTE: To produce blinded version, replace "0" with "1" below.
\newcommand{\blind}{0}

% DON'T change margins - should be 1 inch all around.
\addtolength{\oddsidemargin}{-.5in}%
\addtolength{\evensidemargin}{-.5in}%
\addtolength{\textwidth}{1in}%
\addtolength{\textheight}{1.3in}%
\addtolength{\topmargin}{-.8in}%

%% load any required packages here
\usepackage[french]{babel}\usepackage[intoc, french]{nomencl}


% tightlist command for lists without linebreak
\providecommand{\tightlist}{%
  \setlength{\itemsep}{0pt}\setlength{\parskip}{0pt}}




\begin{document}


\def\spacingset#1{\renewcommand{\baselinestretch}%
{#1}\small\normalsize} \spacingset{1}


%%%%%%%%%%%%%%%%%%%%%%%%%%%%%%%%%%%%%%%%%%%%%%%%%%%%%%%%%%%%%%%%%%%%%%%%%%%%%%

\if0\blind
{
  \title{\bf Rapport de stage}

  \author{
        Fait par Golovanova Elizaveta \\
    BDA, Université Grenoble Alpes\\
      }
  \maketitle
} \fi

\if1\blind
{
  \bigskip
  \bigskip
  \bigskip
  \begin{center}
    {\LARGE\bf Rapport de stage}
  \end{center}
  \medskip
} \fi

\bigskip
\begin{abstract}

\end{abstract}

\noindent%
{\it Keywords:} 
\vfill

\newpage
\spacingset{1.45} % DON'T change the spacing!

\newpage
\tableofcontents 
\newpage

\hypertarget{introduction}{%
\section{Introduction}\label{introduction}}

Ce rapport est consacré à l'analyse du déroulement de stage dans le
cadre de la formation en Master 1, parcours Business et analyse de
données, Faculté d'Economie de l'Université Grenoble Alpes. La première
partie de ce rapport décrira le processus de recherche de mon stage,
l'évolution de mon CV et de ma lettre de motivation, et comment ma
candidature s'etait diffusée à diverses organisations. La deuxième
partie du rapport parlera de l'organisation où j'ai effectué mon stage,
notamment ses principales activités, le nombre d'employés. La troisième
partie est réservée à la description du projet auquel j'ai participé,
c'est-à-dire son objectif global et mon rôle dans sa réalisation. Dans
la quatrième partie, ma mission principale sera décrite en détail. Dans
la cinquième partie finale, les résultats de mon stage seront résumés.

\section{Recherche de stage}
\label{sec:first}

Dans cette section, je décrirai comment mon CV et ma lettre de
motivation ont évoloué en fonction de l'expérience de recherche d'un
stage. Au départ, j'ai rédigé un CV qui ne contenait que des faits nus
sur ma formation, mon expérience de travail et mes compétences. Par
exemple, je n'ai pas décrit mes tâches dans des emplois antérieurs, ni
précisé les matières que j'ai suivies pendant mes études. Au fil du
temps, je me suis rendu compte que ces détails sont importants pour se
démarquer des autres candidats. J'ai réfléchi plus attentivement à mes
avantages, listé mes réalisations au travail, les sujets que j'ai
étudiés. J'ai également ajouté à mon CV mes qualités personnelles que
les recruteurs recherchaient pour le poste d'analyste de données.

J'ai aussi initialement rédigé une lettre de motivation individuellement
pour chaque poste, en précisant le nom de l'entreprise, le nom du
recruteur, pour que la lettre ait l'air personnelle. Cependant, cela a
pris beaucoup de temps et n'a donné aucun résultat. Comme le processus
de candidature en France est administrativement compliqué, avec le
temps, j'ai décidé de simplifier ma lettre de motivation et de la rendre
universelle.

Je tiens également à souligner la participation de l'université aux
modifications qui ont eu lieu. Avec l'aide de Sylvian Houset, les fautes
de frappe et les formulations inexactes ont été corrigées dans mon CV et
ma lettre de motivation.

Après avoir modifié mon CV et ma lettre de motivation, j'ai commencé à
recevoir des offres d'emploi, mais elles ne me convenaient pas pour
poursuivre le processus de formation. J'ai parcouru plusieurs entretiens
et noté les entreprises où je pourrais essayer d'obtenir un emploi après
la fin de ma formation. En parallèle, j'ai diffusé activement mon CV aux
entreprises de Grenoble et ses environs, y compris les offres envoyées
par les responsables de notre master, et je me suis renseignée également
sur les postes disponibles parmi les professeurs de mon université. Au
total, j'ai postulé à plus de 70 endroits pour toute la recherche.
Finalement, j'ai trouvé un stage dans le laboratoire GAEL avec Paolo
Crosetto qui a été chargé du cours d'analyse de données en R que j'ai
suivi. Ayant déjà 4 ans d'expérience dans le domaine scientifique en
Russie et en cherchant la possibilité de poursuivre mon activité dans le
niveau mondial, j'étais convaincue que ce stage s'intègre parfaitement
dans mon projet professionnel.

\section{Présentation de la structure de stage}
\label{sec:second}

L'organisme qui m'a accueilli est l'Institut National de Recherche pour
l'Agriculture, l'alimentation et l'Environnement (INRAE). J'ai effectué
un stage dans un des laboratoires affiliés à cet organisme, à savoir au
laboratoire d'économie appliquée à Grenoble (GAEL). Il est composé d'une
quarantaine de chercheurs auxquels s'ajoutent des post-doctorants, des
doctorants et du personnel administratif et d'appui à la recherche.

\section{Missions effectuées pendant le stage}
\label{sec:third}

La thème de mon stage était ``Méta-analyse de la validité externe des
tâches d'élicitation des risques''. Ma mission principale était de
collecter une importante base de données d'élicitation de risque et de
créer un site web basé sur celle-ci. Pour y parvenir, j'ai dû effectuer
plusieurs tâches différentes. Tout d'abord, j'ai lu de nombreux articles
scientifiques différents basés sur des expériences en laboratoire. Puis,
j'ai traité ces données afin d'obtenir une base de données unifiée et
uniformisée pour toutes les expériences. Ensuite, sur la base des
données reçues, je les ai visualisées en utilisant le langage de
programmation R et le package Shiny.

Les expériences en laboratoire impliquent l'exécution de certaines
tâches visant à identifier l'attitude du répondant face au risque. La
tâche peut être donnée explicitement sous la forme d'une question
directe, par exemple, que «~Comment vous voyez-vous~? Êtes-vous
généralement une personne totalement prête à prendre des risques ou
essayez-vous d'éviter de prendre des risques~?~» (\citet{SOEP2007}) et
indirectement sous la forme d'une loterie, par exemple, où le répondant
est invité à choisir entre deux options plus et moins risquées
(\citet{Holt2002}).

La visualisation sur le site comprenait la conception de la structure du
site, ainsi que son contenu. Par exemple, construire des distributions
pour le paramètre d'aversion relative constante du risque
(CRRA)\footnote{\url{https://en.wikipedia.org/wiki/Risk_aversion}} pour
chaque élément ou chaque type de question, calculer également des
corrélations, creation de cartes géographiques, de tableaux avec des
sources de données, etc.

L'objectif global de ce projet est d'apporter plusieurs améliorations
dans le domaine de l'identification des risques, c'est-à-dire de mieux
comprendre l'état actuel de la validité externe des mesures
d'identification des risques (par méta-analyse). L'un des objectifs
principaux du projet est de rendre l'accès libre aux informations
traitées au cours de l'étude, open source, en ligne et une base de
données en constante augmentation.

Il est à noter, que le projet pré-existait à mon stage, il a été lancé
en 2019 sous la direction de Paolo Crosetto. Au début de mon stage, une
version brute du site était prête, mais elle devait être entièrement
repensée. Lors de la creation d'un nouveau site, j'ai pris quelques
idées conceptuelles de la version originale.

\section{Développement d’un site web consacré d'analyse de risque}
\label{sec:fourth}

Plan :

\begin{itemize}
\item  *Revue de la littérature* sur l'élicitation de risque, y compris des articles avec des tâches d'élicitation de risque, des questionnaires

\item  *Problématique économique* : Est-il possible d'améliorer le pouvoir prédictif des tâches liées à l'elicitation de risque ?

\item *Méthodologie* : explication du contenu ua site web

\item *Résultats* : La partie d'analyse des données que nous n'avons pas encore commencée

\item *Limitations et difficultées*

\end{itemize}

\section{Conclusion}
\label{sec:fifth}

\bibliographystyle{agsm}
\bibliography{bibliography.bib}


\end{document}
