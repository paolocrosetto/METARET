% !TeX program = pdfLaTeX
\documentclass[12pt]{article}
\usepackage{amsmath}
\usepackage{graphicx,psfrag,epsf}
\usepackage{enumerate}
\usepackage{natbib}
\usepackage{textcomp}
\usepackage[hyphens]{url} % not crucial - just used below for the URL
\usepackage{hyperref}

%\pdfminorversion=4
% NOTE: To produce blinded version, replace "0" with "1" below.
\newcommand{\blind}{0}

% DON'T change margins - should be 1 inch all around.
\addtolength{\oddsidemargin}{-.5in}%
\addtolength{\evensidemargin}{-.5in}%
\addtolength{\textwidth}{1in}%
\addtolength{\textheight}{1.3in}%
\addtolength{\topmargin}{-.8in}%

%% load any required packages here
\usepackage[french]{babel}\usepackage[intoc, french]{nomencl}


% tightlist command for lists without linebreak
\providecommand{\tightlist}{%
  \setlength{\itemsep}{0pt}\setlength{\parskip}{0pt}}




\begin{document}


\def\spacingset#1{\renewcommand{\baselinestretch}%
{#1}\small\normalsize} \spacingset{1}


%%%%%%%%%%%%%%%%%%%%%%%%%%%%%%%%%%%%%%%%%%%%%%%%%%%%%%%%%%%%%%%%%%%%%%%%%%%%%%

\if0\blind
{
  \title{\bf Rapport de stage}

  \author{
        Fait par Golovanova Elizaveta \\
    BDA, Université Grenoble Alpes\\
      }
  \maketitle
} \fi

\if1\blind
{
  \bigskip
  \bigskip
  \bigskip
  \begin{center}
    {\LARGE\bf Rapport de stage}
  \end{center}
  \medskip
} \fi

\bigskip
\begin{abstract}

\end{abstract}

\noindent%
{\it Keywords:} 
\vfill

\newpage
\spacingset{1.45} % DON'T change the spacing!

\newpage
\tableofcontents 
\newpage

\hypertarget{introduction}{%
\section{Introduction}\label{introduction}}

Ce rapport est consacré à l'analyse du déroulement de stage dans le
cadre de la formation en Master 1, parcours Business et analyse de
données, Faculté d'Economie de l'Université Grenoble Alpes. La première
partie de ce rapport décrira le processus de recherche de mon stage,
l'évolution de mon CV et de ma lettre de motivation, et comment ma
candidature s'etait diffusée à diverses organisations. La deuxième
partie du rapport parlera de l'organisation où j'ai effectué mon stage,
notamment ses principales activités, le nombre d'employés. La troisième
partie est réservée à la description du projet auquel j'ai participé,
c'est-à-dire son objectif global et mon rôle dans sa réalisation. Dans
la quatrième partie, ma mission principale sera décrite en détail. Dans
la cinquième partie finale, les résultats de mon stage seront résumés.

\section{Recherche de stage}
\label{sec:first}

Dans cette section, je décrirai comment mon CV et ma lettre de
motivation ont évoloué en fonction de l'expérience de recherche d'un
stage. Au départ, j'ai rédigé un CV qui ne contenait que des faits nus
sur ma formation, mon expérience de travail et mes compétences. Par
exemple, je n'ai pas décrit mes tâches dans des emplois antérieurs, ni
précisé les matières que j'ai suivies pendant mes études. Au fil du
temps, je me suis rendu compte que ces détails sont importants pour se
démarquer des autres candidats. J'ai réfléchi plus attentivement à mes
avantages, listé mes réalisations au travail, les sujets que j'ai
étudiés. J'ai également ajouté à mon CV mes qualités personnelles que
les recruteurs recherchaient pour le poste d'analyste de données.

J'ai aussi initialement rédigé une lettre de motivation individuellement
pour chaque poste, en précisant le nom de l'entreprise, le nom du
recruteur, pour que la lettre ait l'air personnelle. Cependant, cela a
pris beaucoup de temps et n'a donné aucun résultat. Comme le processus
de candidature en France est bureaucratiquement compliqué, avec le
temps, j'ai décidé de simplifier ma lettre de motivation et de la rendre
universelle.

Je tiens également à souligner la participation de l'université aux
modifications qui ont eu lieu. Avec l'aide de Sylvian Houset, les fautes
de frappe et les formulations inexactes ont été corrigées dans mon CV et
ma lettre de motivation.

Après avoir modifié mon CV et ma lettre de motivation, j'ai commencé à
recevoir des offres d'emploi, mais elles ne me convenaient pas pour
poursuivre le processus de formation. J'ai parcouru plusieurs entretiens
et noté les entreprises où je pourrais essayer d'obtenir un emploi après
la fin de ma formation. En parallèle, j'ai diffusé activement mon CV aux
entreprises de Grenoble et ses environs, y compris les offres envoyées
par les responsables de notre master, et je me suis renseignée également
sur les postes disponibles parmi les professeurs de mon université. Au
total, j'ai postulé à plus de 70 endroits pour toute la recherche.
Finalement, j'ai trouvé un stage dans le laboratoire GAEL avec Paolo
Crosetto qui a été chargé du cours d'analyse de données en R que j'ai
suivi. Ayant déjà 4 ans d'expérience dans le domaine scientifique en
Russie et en cherchant la possibilité de poursuivre mon activité dans le
niveau mondial, j'étais convaincue que ce stage s'intègre parfaitement
dans mon projet professionnel.

\section{Présentation de la structure de stage}
\label{sec:second}

L'organisme qui m'a accueilli est l'Institut National de Recherche pour
l'Agriculture, l'alimentation et l'Environnement (INRAE). J'ai effectué
un stage dans un des laboratoires affiliés à cet organisme, à savoir au
laboratoire d'économie appliquée à Grenoble (GAEL). Il est composé d'une
quarantaine de chercheurs auxquels s'ajoutent des post-doctorants, des
doctorants et du personnel administratif et d'appui à la recherche.

\section{Missions effectuées pendant le stage}
\label{sec:third}

L'une des principales prémisses de la plupart des modèles
microéconomiques classiques est la rationalité des individus. Cependant,
dans le monde réel, on peut observer que les gens, lorsqu'ils sont
exposés à l'incertitude, essaient de minimiser cette incertitude autant
que possible. Par exemple, un investisseur peut choisir d'investir son
argent dans un compte bancaire avec un taux d'intérêt faible mais
garanti, plutôt que dans des actions, qui peuvent avoir un rendement
attendu élevé mais également comporter un risque de perte. Cet effet est
appelé aversion au risque, il caractérise la propension des personnes à
prendre telle ou telle décision face au risque. L'analyse de l'aversion
au risque joue un rôle important en économie comportementale, car elle
permet de mieux comprendre le fonctionnement de l'irrationalité humain.

La thème du projet où j'ai participé pendant mon stage était
``Méta-analyse de la validité externe des tâches d'élicitation des
risques''. Il vise à rassembler les connaissances existantes dans la
domaine de l'elicitation des risques, de mieux comprendre l'état actuel
de la validité externe des mesures qui aident d'identifier des risques
(par méta-analyse) et donner libre accès à d'autres chercheurs et
personnes intéressées en ligne et avec une base de données en constante
augmentation.

L'enjeu principal de ce projet est de repenser les tâches d'élicitation
des attitudes face au risque. С'est-à-dire qu'au cours d'expériences en
laboratoire, il est censé effectuer certaines tâches visant à identifier
l'attitude du répondant face au risque. Une question directe peut être
posée pour le déterminer, par exemple, que «~Comment vous voyez-vous~?
Êtes-vous généralement une personne totalement prête à prendre des
risques ou essayez-vous d'éviter de prendre des risques~?~»
(\citet{SOEP2007}) ou il peut être la demande d'indiquer la probabilité
d'être impliqué dans un événement particulier, comme tricher à un examen
(\citet{Blais2006}). De même il est possible d'estimer les attitudes
face au risque en utilisant une tâche d'élicitation des risques (RET),
par exemple, où le répondant est invité à choisir entre deux options
plus et moins risquées (\citet{Holt2002}) ou à choisir un seul lotterie
de la liste proposé (\citet{Eckel2002}). Il existe de nombreuses
variantes de ces tâches, certaines impliquent une composante visuelle
pour une perception plus facile (\citet{Hunt2005},
\citet{Crosetto2013}). Dans quelques tâches, les pertes sont implicites
(\citet{Menkhoff_Sakha_2017}), dans d'autres seulement les gains
(\citet{Eckel2002}, \citet{Holt2002}). Dans certaines tâches, il y a le
choix entre deux loteries, dont l'une est plus risquée. D'autres peuvent
offrir un choix de loterie à risque et un certain équivalent
(\citet{Menkhoff_Sakha_2017}, \citet{Csermely2014}).

Le fait est que les RET montrent peu de corrélation avec les mesures
autodéclarées, avec le monde réel et entre eux. En termes
psychométriques, ils montrent peu de validité prédictive. Ainsi, le
résultat de l'analyse des données recueillies au cours du projet devrait
être la creation d'une nouvelle tâche qui résoudrait le problème de la
faible corrélation.

Il est à noter, que le projet pré-existait à mon stage, il a été lancé
en 2019 sous la direction de Paolo Crosetto. Au début de mon stage, une
version brute du site web avec la base de données modérée et les
fonctions limités était prête, mais elle devait être entièrement
repensée. En outre, lors de la creation d'un nouveau site web, j'ai pris
quelques idées conceptuelles de la version originale.

Ma mission principale était de redevelopper un site web informatif, bien
structurisé, techniquement stable et jolie sur logiciel R Studio. Pour y
parvenir, j'ai dû effectuer plusieurs sous-tâches. Tout d'abord,
nombreux articles scientifiques différents basés sur des expériences en
laboratoire ont été lus. Deuxièmement, une demande de données a été
faite pour enrichir le site auprès de la communauté scientifique en
suivant deux methodes :

\begin{enumerate}
\item Paolo a envoyé le mail au ces collegues directement en utilisant la base de contact qu'il avait;
\item Les messages personnels à diffuser aux auteurs qui Paolo ne connait pas encore ont été composé. Les articles neessaires ont été selectionées par definition des requets pertinentes sur Google Scholar. Les liens sur ces articles et son information principale, tel que les noms d'auters et d'article et l'année de publication ont été collecté en utilisant le parser en Python. Ensuite, la lecture des articles par le site Scihub et extraction des mails par analyseur en Python ont été effectué. 
\end{enumerate}

Ensuit, il fallait d'alimenter le site web d'une quantité importante de
base de données d'élicitation de risque qu'on a reçu dans une manière
unifiée et uniformisée pour toutes les expériences. Principalement ça
veut dire de faire les calculs nessecaires pour obtenir le paramètre
d'aversion relative constante du risque (CRRA)\footnote{\url{https://en.wikipedia.org/wiki/Risk_aversion}}
pour chaque article en utilisant la methodologie speciale qui sera
precisé dans la partie 5. Il était également nécessaire d'identifier
sans ambiguïté d'autres variables, telles que le genre, l'age, le pays
dans lequel l'expérience a été menée, la ville, etc.

Finalement, en ulilisant cette base des données, le site web a été
recréé avec l'ajout de nouvelles fonctionnalités en utilisant le package
Shiny. La visualisation sur le site comprenait la conception de la
structure du site, ainsi que son contenu. Par exemple, construire des
distributions pour le paramètre CRRA pour chaque élément ou chaque type
de question, calculer également des corrélations, creation de cartes
géographiques, de tableaux avec des sources de données, etc.

\section{Développement d’un site web consacré d'analyse de risque}
\label{sec:fourth}

\subsection{Problématique économique}

Des tentatives pour faire une méta-analyse des résultats de diverses
expérimentations contenant de tâches d'élicilation du risque ont déjà
été faites dans la littérature (\citet{CroFil2013b}). Cependant, une
analyse détaillée des caractéristiques des tâches, ainsi que des
questions supplémentaires aux répondants sur un large échantillon de
données, n'ont pas été réalisées auparavant. La problématique économique
qui décrit une de mes missions est une recherche des caractéristiques de
la tâche d'élicitation du risque et les questions supplémentaires qui
expliquaient le mieux l'attitude réelle face au risque.

\subsection{Base de données}

Avant mon stage, la base de données était composée d'environ 55 articles
contenant 6 types de RET et 9 questions supplémentaires. Examinons de
plus près les types des tâches. L'une des RET les plus populaires est la
liste de prix multiple de Holt and Laury (HL, \citet{Holt2002}). Les
sujets sont confrontés à une série de choix entre des paires de
loteries, l'option A étant plus sûre que l'option B (Tableau 1).
L'espérance mathématique est commun à tous les choix, et les paires de
loterie sont classées par valeur attendue croissante. Les sujets font un
choix pour chaque paire de loteries et doivent à un moment donné passer
à l'option risquée. Le point de commutation capte l'aversion au risque
du sujet. Un sujet neutre au risque devrait commencer par l'option A et
passer à B à partir du cinquième choix. À la fin de l'expérience, une
ligne est choisie au hasard pour le paiement, et la loterie choisie est
jouée pour déterminer le gain. Il convient de noter que si le répondant
passait de la loterie A à la loterie B, puis vice versa, ces cas étaient
supprimés de la base de données, car le répondant était considéré comme
incohérent. De plus, il peut parfois y avoir un nombre de lignes autre
que 10, par exemple 5 (\citet{Branas_Garza2021}).

\begin{figure}
\centering
\includegraphics[width=0.6\textwidth,height=0.3\textheight]{/Users/elizavetagolovanova/Dropbox/METARET/METARET_github/Report de stage/HL.png}
\caption{La liste de prix multiple de Holt and Laury}
\end{figure}

Il est à noter, que pour calculer le paramètre CRRA pour chaque tâche la
fonction d'utilité \(u(x) = x^r\) a été utilisée. Pour chaque moment de
changement des préférences de la loterie A à la loterie B, il est
possible de déterminer l'intervalle où tombe la valeur de ce paramètre.
Par exemple, si le répondant a choisi sur la quatrième ligne la loterie
B, alors la solution de l'équation
\(0.4*4^r + 0.6*3.2^r = 0.4*7.7^r + 0.6*0.2^r\) sera le seuil supérieur
de l'intervalle et la solution de
\(0.3*4^r + 0.7*3.2^r = 0.3*7.7^r + 0.7*0.2^r\) sera celui de inférieur.
Lors de la mise à jour du site, pour ce répondant une des valeurs de
CRRA distribué uniformément est implémentée avec une valeur minimale de
0.32 et une valeur maximale de 0.59. Comme ça les valeurs appropriées
ont été calculées pour tous les participants de ce type de tâche.

Dans le point de commutation sur la toute première loterie, seul le
seuil inférieur est utilisé, et dans le cas de la dixième - seul le
seuil supérieur. Si le répondant a choisi la loterie sûre tout le temps,
alors de tels cas ont été exclus.

Une loterie pas peu connue est aussi la sélection de loteries triées de
Eckel and Grossman (EG, \citet{Eckel2002}). Les sujets choisissent une
loterie parmi un ensemble de 5 loteries caractérisées par une valeur
attendue croissante linéairement ainsi qu'un écart type plus grand, avec
la probabilité de chaque option étant fixée à 50 \% (Tableau 2). Un
participant neutre au risque devrait choisir la loterie 5, car elle
donne la valeur attendue la plus élevée.

\begin{figure}
\centering
\includegraphics[width=0.6\textwidth,height=0.3\textheight]{/Users/elizavetagolovanova/Dropbox/METARET/METARET_github/Report de stage/EG.png}
\caption{La sélection de loteries triées de Eckel and Grossman}
\end{figure}

De manière similaire à la procédure de calcul du CRRA de la tâche
précédente, pour chaque loterie sélectionnée, les seuils supérieur et
inférieur de l'intervalle sont calculés. Par exemple, si 3 loterie est
sélectionné, alors le seuil supérieur sera la solution d'équation
suivant \(8^r + 2^r = 10^r + 1^r\).

Plan :

\begin{itemize}


\item *Méthodologie* : explication du contenu ua site web


\item *Résultats* : La partie d'analyse des données que nous n'avons pas encore commencée

\item *Limitations et difficultées*

\end{itemize}

\section{Conclusion}
\label{sec:fifth}

\bibliographystyle{agsm}
\bibliography{bibliography.bib}


\end{document}
